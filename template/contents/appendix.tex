% !Mode:: "TeX:UTF-8"

\chapter{Derivation of Back Propagation}

\chapter{Code of This Work}
The source code is available in my GitHub
(\url{https://github.com/CrazyRundong/deep-stress}).
For thesis format requirements, code are also attached here.

\section{Data Preprocessing}
\begin{minted}[mathescape, linenos, breaklines]{python}
# -*- coding: utf-8 -*-
import mxnet as mx
import numpy as np
from scipy.interpolate import griddata
from datetime import datetime
import os
import re
from sklearn.model_selection import train_test_split

comsol_dir = './Data/comsol_source'
new_data_dir = './Data/comsol_new'
data_dirs = [comsol_dir, new_data_dir]

data_path = './Data/comsol_crops.npz'
vst_num = 12  # num of vst per axis
crop_size = 224  # follow standard CNN input size
radius_scale_factor = 0.2  # R = grid_size * radius_scale_factor
n_points = int(crop_size * (vst_num + 1) / radius_scale_factor)  # num of points interpolated per axis
grid_x, grid_y = np.mgrid[0:1:complex(0, n_points), 0:1:complex(0, n_points)]
circular_div = 16  # K in paper
feat_num = 500  # num of feature extracted per sample
plot_num = 20  # num of test samples to plot


def load_dir(dirs=data_dirs):
    assert isinstance(dirs, list)
    name_pattern = r'([a-zA-Z]+)(\d+)'
    temp_paths = []
    stress_paths = []
    
    for data_dir in dirs:
        assert os.path.exists(data_dir)
        for root, _, files in os.walk(data_dir):
            for fname in (os.path.splitext(file)[0] for file in files):
                g = re.match(name_pattern, fname, re.I)
                if g:
                    if g.group(1) == 'stress':
                        continue
                    else:
                        tem_path = os.path.join(root, 'tem' + g.group(2) + '.txt')
                        stress_path = os.path.join(root, fname + '.txt')
                else:
                    # old data names
                    fname = os.path.splitext(fname)[0]
                    tem_tokens = fname.split('_')
                    if tem_tokens[0] == 'stress':
                        continue
                    tem_path = os.path.join(root, fname + '.txt')
                    stress_name = '_'.join(['stress'] + tem_tokens[1:]) + '.txt'
                    stress_path = os.path.join(root, stress_name)
                assert os.path.exists(stress_path) and os.path.exists(tem_path)
                temp_paths.append(tem_path)
                stress_paths.append(stress_path)
    return temp_paths, stress_paths


def load_and_crop(data_set_dir=data_dirs):
    vst_locate = np.linspace(0., 1., vst_num + 2)[1: -1]
    radius = 1. / (vst_num + 1) / 2. * radius_scale_factor
    tem_list = []  # the X
    stress_list = []  # the y
    temp_paths, stress_paths = load_dir(data_set_dir)
    for tem_path, stress_path in zip(temp_paths, stress_paths):
        # skip rows with Chinese character
        tem_data = np.loadtxt(tem_path, usecols=(0, 1, 3), dtype=np.float32, comments='%', skiprows=8)
        stress_data = np.loadtxt(stress_path, usecols=(0, 1, 3), dtype=np.float32, comments='%', skiprows=8)
        # scan to [0, 1]
        tem_data[:, :2] *= 100
        stress_data[:, :2] *= 100
        for x in vst_locate:
            for y in vst_locate:
                # TODO(Rundong): not Pythonic, I gona crazy by this, believe me...
                current_tem = tem_data[
                    np.logical_and(np.logical_and((x - radius) <= tem_data[:, 0], tem_data[:, 0] <= (x + radius)),
                                   np.logical_and((y - radius) <= tem_data[:, 1], tem_data[:, 1] <= (y + radius)))]
                current_tem[:, :2] -= (x, y)
                current_tem[:, :2] /= 2 * radius

                current_stress = stress_data[
                    np.logical_and(np.logical_and((x - radius) <= stress_data[:, 0], stress_data[:, 0] <= (x + radius)),
                                   np.logical_and((y - radius) <= stress_data[:, 1], stress_data[:, 1] <= (y + radius)))]
                stress_max = current_stress.max()
                tem_list.append(current_tem)
                stress_list.append(stress_max)

    return tem_list, stress_list


def local_interp(tem_list, stress_list, npz_dir=data_path):
    interp_size = 28  # follow MNIST
    xx, yy = np.mgrid[0:1:complex(0, interp_size), 0:1:complex(0, interp_size)]
    interped_list = []
    for tem in tem_list:
        tem[:, :2] += 0.5  # norm (x, y) to [0, 1]
        c = griddata(tem[:, :2], tem[:, -1], (xx, yy), method='cubic', fill_value=tem[:, -1].mean())
        interped_list.append(c)
    interped_list = np.array(interped_list)
    stress_list = np.array(stress_list)

    # Train / Val split
    temp_train, temp_val, stress_train, stress_val = train_test_split(interped_list, stress_list, train_size=0.7)

    tem_mean = np.mean(temp_train, axis=0)
    tem_var = temp_train.var()

    temp_train -= tem_mean
    temp_val -= tem_mean
    temp_train /= tem_var
    temp_val /= tem_var

    tem_max = np.abs(temp_train).max()

    stress_max = stress_train.max()
    stress_train /= stress_max
    stress_val /= stress_max

    # Dump npz
    np.savez_compressed(npz_dir,
                        temp_train=temp_train,
                        temp_val=temp_val,
                        stress_train=stress_train,
                        stress_val=stress_val,
                        max_temp=tem_max,
                        max_stress=stress_max)


def generate_mx_array_itr(data_, label_, batch_size_=10, shuffle_=True):
    assert data_.shape[0] == label_.shape[0]
    n, h, w = data_.shape
    itr = mx.io.NDArrayIter(data_.reshape((n, 1, h, w)), label_, batch_size_, shuffle=shuffle_, label_name='reg_label')
    return itr


def main(npz_path=data_path):
    print('{}: Loading data...'.format(datetime.now().strftime('%Y.%m.%d-%H:%M:%S')))
    tem_list, stress_list = load_and_crop()
    print('{}: Crop and interpolating data...'.format(datetime.now().strftime('%Y.%m.%d-%H:%M:%S')))
    local_interp(tem_list, stress_list, npz_path)
    print('{}: Data dump done.'.format(datetime.now().strftime('%Y.%m.%d-%H:%M:%S')))

if __name__ == '__main__':
    main('./Data/comsol_new.npz')

\end{minted}
