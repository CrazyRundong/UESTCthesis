% !Mode:: "TeX:UTF-8"

\chapter{\centering{3D Interconnection and Packaging:} \protect \\
    \centering{Impending Reality or Still a Dream?}}

As integrated circuit technology continuously improves its integration
capabilities, and SoC is a possibility, the limitations of 2D-interconnects
are more obvious. Integrated circuits are partioned in function
blocks (``tiles'') that are interconnected using routing in parallel
planes (2D). With the scaling of the technology, these long lines
have become the speed-limiting factor on the chip, as well as a significant
source for power consumption. If the functional ``tiles'' on the
chip could be stacked in three dimensions, the chip area would be
reduced and much shorter interconnects between would result.

Current developments in packaging technology, originally developed
in the field of MEMS technology, are delivering the key
enabling technologies for building 3D stacked devices. These are:
\begin{itemize}
    \item Thinning of wafers, below 50μm and down to the active
    layer.
    \item Wafer-to-wafer bonding techniques, up to 300mm diameter
    wafers.
    \item Die-to-wafer bonding
    \item Wafer-through-hole technologies: electrically isolated connections
    through $Si$.
\end{itemize}

The 3D interconnects may be realized at different levels of the
interconnect hierarchy. They can be realized at the chip-level,
interconnecting the ``traditional'' pins of the die in the third
dimension. This is defined as a 3D-``system-in-a-package'' (3D-SIP)
approach. The interconnectivity in the third dimension is
rather poor in this approach as connections pass through ``package
pins''. The available 3D-interconnectivity between two layers is
typically only $2-5/mm$ along the perimeter of the device or $4-9/mm^2$
when area array packages are used. Due to these limitations, this
approach is limited to stacking of dies with limited pin count and
limited interconnectivity between the die. Each layer of the stack
is a ``sub''-system on a chip or system-in-a-package.

Many systems require a much higher interconnectivity between
each die in the stack. Circuit blocks (``tiles'') within a die need
interconnections to circuit blocks on another die, without passing
through the regular chip I/O pads. These interconnections must
behave as on-chip ``global wiring'' interconnects and thus have low
resistance and capacitance, allowing for fast interconnects and/or
low power interconnectivity. This is defined as a 3D-``system-on-a-chip'' (3D-SoC). 
One particular potential example is the stacking
of a memory die on a logic IC, where logic-``tiles'' are directly
addressing smaller memory banks on the memory die, without
going through the central bus. This requires a 3D interconnectivity,
in the order of $100/mm^2$.

Finally the 3D interconnects are envisaged at the lowest level,
the transistors themselves. In this case, the technology should
be considered a ``scaling-driven'' technology, integrating more
transistors per unit area. The 3D interconnects are only local
interconnects. The global on-chip interconnect layers on chip
remain essentially 2D interconnect layers. This is defined as a
``3D-integrated circuit'' technology (3D-IC).

In order to be successful, 3D-SIP, 3D-SoC or 3D-IC technologies
need to become manufacturable technologies. In particular a
high yield has to be reached. Preferably, individual layers are
processed in parallel, tested and finally, stacked as ``known-good-devices''.
This goal is likely to be reached first for 3D-SIP, followed
by 3D-SoC. It is much further out for 3D-IC technologies.

For the realization of a 3D-SIP, a relatively low interconnect density
in the third dimension is available. Each layer should therefore
consist of a well-defined sub-circuit block, using SoC-technology
to enable a small sized solution. These components,
which may be individual SIP packages are preferably pre-tested
(``known-good-SIP'') to ensure a high final assembly yield.

An example of a 3D-SIP integration scheme, realized at IMEC, is
a fully integrated low power rf transceiver shown in --.
This device measures only $7 \times 7 mm$ and consists of two CSP-type
devices (CSP=``Chip-Scale-Package''). The top CSP is realized
using IMEC's rf-MCM-D technology with integrated passives.[1]
The bottom CSP is a double-sided high-density printed circuit
board with a high density flip chip die on the bottom side and
several discrete passive components mounted on the topside.
The connection of this bottom part to the top part is obtained by
using solder balls on the topside of the bottom laminate and
encapsulation of the topside devices.

One example of a 3D-SoC technology is the, ``ultra thin chip stacking''
technology (UTCS), under development at IMEC.[2] A
schematic build-up is shown in Figs. 7.4.2 and 7.4.3. In this structure,
IC's, thinned down to $10-15 \mu m$, are embedded inside a multilayer
thin film build-up, realized on a base CMOS wafer. This
requires a technique to thin and transfer the active die. An example
of an embedded die is shown in Fig. 7.4.4.

By using thin film lithography, a very high number of interconnects
can be realized between dies. This allows for additional
interconnects between sub-sections of the die. In the case of the
proposed UTCS structure, via connections in the third dimension
are realised in the area around the chips. As these thin film vias
are realized with pad sizes smaller than $50mm$, a very high
interconnect density in the third dimension is obtained.

For the realization of such 3D-SoC structures, very aggressive
wafer and die thinning is used, eventually reducing the die to the
active circuit layer. The use of very thin layers is improves the 3D-connectivity
as it relaxes the required aspect ratios for these structures.
It is also important for the thermal performance of the stack.
The heat flow in such a structure is considered to be almost one dimensional.
The heat flows vertically through the different layers
of the stack with little lateral spreading. For the UTCS structure,
lateral heat flow is limited to about $100-200mm$. The ``package'' of
the 3D-SoC evauates the heat to the ambient, similar to traditional
packages. A further limiting factor to the allowable heat dissipation
of the stack is its internal thermal resistance. To realize such
stacks, thin polymeric dielectric films are used as ``glue'' – layers,
which dominate the thermal performance of the stack. The thermal
resistance of $1mm$ polymer is about equivalent to $10\mu m \; SiO_2$
(BEOL), to $1mm$ of $Si$ or to $2mm$ of copper. For structures similar
to the one shown in Fig. 7.4.3, a thermal interface resistance of $0.3$
to $0.75 K/W$ was measured ($0.5 K/W = \textrm{a} 10\mu m$ thick polymer (BCB)
interface layer).To reduce the thermal resistance of the sandwich
layer, vertical metal structures are used as they conduct heat much
more efficiently ($Cu$ is $2\times$ better than $Si$). They are, however, only
efficient if they are realized as via connections through the die area.

Also of major concern is the electrical integrity of the transferred
and transplanted die. In order to investigate this issue, $16 \; 20\times20
mm\;0.35 \mu m$ CMOS test die were measured before and after thinning
to $45 \mu m$ (wafer level) and after further thinning to $15 \mu m$ and
transfer to a host substrate. Some typical results are shown in figure
5 and 6. Given the proper process conditions, standard CMOS
die can be thinned to very thin layers without yield loss.

3D interconnection and packaging is one of the most active R\&D
topics. Enabling technologies for realizing 3D interconnects
coincide with a growing need for heterogeneous system integration。