% !Mode:: "TeX:UTF-8"

\chapter{3D 集成与封装:呼之欲出还是遥遥无期?}

随着集成电路的技术革新,其集成度被逐步提高,与此同时 2D 集成的弊端也愈发明显。
根据功能不同,集成电路被分为由 2D 线路连接的独立并行模块(tiles)。
随着集成度的提高,这些连线成为了限制芯片速度和产生额外功耗的因素之一。
如果芯片上的这些功能模块能够在三维空间进行堆叠,芯片面积将得到有效控制,
与此同时线路长度也能够显著缩短。

当前封装技术,特别是 MEMS 领域技术的进展,为构建 3D 堆叠器件提供了关键技术支撑。
这些技术是:
\begin{itemize}
    \item 厚度低于 $50 \mu m$,直达激活层的薄晶圆(wafer)
    \item 最大可达 $300 mm$ 的晶圆-晶圆连接技术
    \item 晶圆-裸片(die)连接技术
    \item 晶圆穿孔连接:绝缘的穿 $Si$ 连接
\end{itemize}

3D 连接技术可在不同的互联层级实现。
它可以在芯片级别,在垂直维度联结裸片的针脚(pin)。
这被定义为“3D系统封装(3D-System-in-a-package, 3D-SIP)”技术。
该技术中的垂直互联非常稀少,仅存在于封装针脚之间。
可用的垂直连接密度通常为器件边界 $2-5\textrm{连接}/mm$,或是在所用封装面积上
 $4-9\textrm{连接}/mm^2$。
由于这些限制,这类技术仅被用于有限针脚和有限连接的裸片堆叠。
堆叠的每一层可看作是芯片,或被称为“封装系统”的一个子系统。

很多系统要求堆叠的裸片之间有更多的连接。
一个裸片中的电路模块(tiles),要在不穿过芯片传统 I/O 引脚的前提下,
连接至另一个裸片中的电路模块。
这些连接必须能够作为片上“全局连线”,有较低的阻抗和容抗,以确保连接的速率和低功耗。
这被定义为“3D 片上系统(3D-System-on-a-chip,3D-SoC)”技术。
一个重要的潜在应用是,在不占用总线的情况下,将一系列存储裸片堆叠在一片逻辑芯片内,
其中的逻辑模块(dies) 是存储裸片的地址单元。
这要求超过 $100\textrm{连接}/mm^2$ 的 3D 连接密度。

最终,3D 连接达到了最底层——晶体管级别连接。
在此层级下,该技术成为了一种以尽可能提高单位面积晶体管密度为目标的“缩放驱动”的技术。
其中的 3D 连接都是局部连接。
芯片中所必须的全局连接层仍维持了 2D 连接。
这被定义为“3D 集成电路(3D-Integrated-circuit,3D-IC)”技术。

为了确保成功,3D 系统封装、3D 片上系统、3D 集成电路,都用该是能够量产的技术,以获得较大利润。
最好的情况是,各个独立层都应该是独立运行、测试和封装的“验证过的良品”。
这个目标很可能按照 3D 系统封装、3D 片上系统、3D 集成电路的顺序,逐一实现。

为实现 3D 系统封装,与之相关的垂直维度低密度互联技术必须是可用的。
因此,系统中的每一层都应该被保持为定义良好的子电路模块,同时用片上系统技术缩减芯片面积。
这些元件(也可能是独立的系统封装),应该是经过完善测试的系统封装,以确保整个系统的高效结合。

IMEC 实现了一个低功耗完全集成的 RF 转换器,如图 TODO 所示。这是一个 3D 系统封装的例子。
该器件面积仅为 $7 \times 7 mm$,由两层芯片缩放封装(Chip-Scale-Pakage,CSP)组成。
顶层 SCP 由 IEMC 的配有积化被动器件的 rf-MCM-D 技术所实现。
底层 SCP 是配有高密度底层翻转芯片裸片(flip chip die) 和若干顶层离散积化元件的高密度印刷电路板。
顶层和底层间的连接,由底层金属薄板顶部和顶层器件间的焊接球实现。

3D 片上系统的一个例子是 IMEC 正在开发的“超薄芯片堆叠”(ultra thin chip stacking,UTCS)技术。
图 TODO 简略展示了其构造。
在此结构中,薄至 $10-15\mu m$ 的 IC,在 CMOS 晶圆上多层堆叠集成。
这需要做薄和转移激活裸片的技术。
裸片堆叠的一个例子是图片 TODO。

通过薄底片光刻技术,裸片间可实现相当高的互联密度。
这使得裸片的子部分互联成为可能。
在提出的 UTCS 结构中,垂直维度的穿孔互联都存在于芯片边缘区域。
由于这些穿孔都由小于 $50mm$ 的 pad 实现,我们可以获得高密度的垂直维度互联。

为了实现 3D 片上系统结构,相当激进的薄化晶圆和裸片技术被提了出来,甚至包括将裸片做进激活层的技术。
由于减缓了所需的宽长比,使用极薄的层可以改善 3D 连接性。
这也可以改善 3D 堆叠的热效应。
热流通常被认为是一维结构。
在 UTCS 结构中,侧向热流通常被限制为 $100-200 mm$。
3D 片上系统的封装将限制热流导至周围环境中,和传统封装相类似。
另一个限制堆叠散热的因素,是堆叠内部的热阻抗。
为了实现这样的堆叠,薄的聚合物介电膜被当做“胶水”使用——这会影响堆叠的散热效果。
$1mm$ 聚合物的热阻抗近似于 $10 \mu m \; SiO_2$(BEOL),或是 $1mm \; Si$ 或 $2mm \; Cu$。
在类似于图 TODO 中的结构中,我们测得的热界面阻抗约为 $0.3-0.75K/W$
($0.5K/W$ 等价于一个 $10 \mu m$ 厚度的聚合物界面层)。
为减小这些界面层的热阻抗,垂直金属结构被大量采用,因为其导热系数较大($Cu$ 是 $Si$ 的二倍)。
然而,只有在裸片上的穿孔连接处实现这样的结构,才是有效的。

当然,另一个主要的关切是转移和移植的裸片的电气完整性。
为了解决这个问题,$16$ 片 $20 \times 20 mm$ 的 $0.35 \mu m$ CMOS 晶圆,
在厚度缩减至 $45\mu m$ 和 $15 \mu m$ 以转移至另一主衬底之前和之后,分别进行了测量。
一些典型结果展示在图片 TODO 中。
在确定的处理条件下,标准 CMOS 裸片可以在不损失功能正确性的情况下被做得很薄,以实现堆叠。

3D 互联和封装是最活跃的研发领域之一。实现 3D 互联技术的发展也伴随着各种系统集成的需求。
