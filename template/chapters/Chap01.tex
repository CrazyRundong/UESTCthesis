% !Mode:: "TeX:UTF-8"

\chapter{Introduction}
Three-Dimensional Integrated Circuits (3D ICs),
which consists of multiple vertically stacked dies, 
are emerging as a natural way to overcome interconnect scaling problems in 2D ICs.
While exploring the benefits such as enabling heterogeneous integration, 
alleviating interconnect barrier problem, and improving power performance, 
3D ICs suffers severe thermal induced reliability problems due to its
stacked structure's poor heat dissipation potential.

Vertically stacked dies in 3D ICs are connected via $Cu$ Though Silicon Vias (TSV)\cite{Beyne2008Through}. 
Due to the dramatic variance of 
\textit{coefficient of thermal expansion} (CTE) and \textit{thermal conductivity}
of $Cu$ and $SiO_2$ on the metal-silicon connection of TSVs, 
wafers will encounter with lager thermal stress near each TSV.
This worsen the runtime reliability of 3D ICs in heavy-load situations.

Dynamic thermal management (DTM) techniques are proposed to address these reliability
problems. But due to lacking of information on the reliability side, existing DTM 
methods may only minimize the average temperature of hole chip and reduce temperature
variance over difference parts on the IC body\cite{Zou2013Thermomechanical}.
However, the thermal stress around TSV strongly depends on temperature gradient rather
than the na\"ive temperature value, only reducing the average temperature may not help with
the run time reliability \cite{Zhang2016Fast}.

To accurately simulate the thermal behavior around TSVs, we can apply finite element methods (FEM)
based models to predict the thermal stress from thermal distribution \cite{Lu2009Thermo}. 
However, due to its high computing costs, FEM based models may not be used in runtime scenarios.

Based on artificial neural networks (ANN) and large amount of existing FEM generated data, 
\cite{Zhang2016Fast} proposed a fast runtime reliability analysis method to predict the 
thermal stress near TSVs from their thermal distribution on the
runtime, which achieved reasonable accuracy under economic computing costs.
However, this method gets two main drawback: one have to apply hand-crafted feature extraction to
thermal data, which makes the train-predict process not end-to-end, and the hand-crafted feature used here
could be complicated and not reliable. Another is, 
for the lacking of capability of shallow ANN model used in this method, 
the accuracy is not satisfied enough.

In this work, we propose a modified runtime reliability analysis method.
Based on deep convolutional neural networks (CNN), the model used in our method can extract features from raw
thermal data automatically, makes the train-prediction into a end-to-end process.
While maintaining a economic computing cost, our method achieves more than $2\times$ accuracy to \cite{Zhang2016Fast}.

Structure of this work: In chapter \ref{chap::pre}, we introduce some basic conceptions about 3D IC and neural
networks. In chapter \ref{chap::Model}, we briefly review the work of \cite{Zhang2016Fast}, demonstrate
our proposed deep learning based models, and report their experimental performance.
In chapter \ref{chap::conc}, we summarize our work in this paper, and give prospection to future work.
