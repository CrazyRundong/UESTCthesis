% !Mode:: "TeX:UTF-8"

\chapter{Deep Learning Based Model for Runtime Analysis}
In this chapter we describe our proposed deep learning based method for 3D IC runtime analysis.
First, we briefly review the work of \cite{Zhang2016Fast}, for their model
will be partially used in our work.
Then we describe the models used in our proposed method. They are modified from some modern
CNNs used in computer vision jobs such as object detection and face recognition.
We will specify these models' structure, and how they differ from their inceptions
to be fitted into 3D IC reliability runtime analysis jobs.
Finally, we apply this method to the dataset in \cite{Zhang2016Fast},
get a promising accuracy and economic computing cost.
The details of this experiment are demonstrated in the last section.

\section{A Fully-Connected ANN Model}
Since FEM based models are computational expensive, we may wish to figure out whether
it exist a much simpler and faster map $f: T \to F_{\left\{ x,y,z \right\}}$
\footnote{$T$ and $F_{\left\{ x,y,z \right\}}$ are notations in section \ref{sec::thermal}, which means
$f$ is a map from IC temperature distribution to its relevant thermal stress}
that could be used in runtime scenario.
\cite{Zhang2016Fast} proposed a two-layers fully connected ANN model with hand-crafted
feature extraction, then train this model by large amount of $(T \to F_{\left\{x,y,x\right\}})$ data
generated off-line by FEM based model. 
The ANN model achieved accuracy of $RMSE: 0.0779$ in the normed test set, and fast enough 
($300 TSVs / 0.06s$ @ 2.4 GHz, 1 core) to be applied
in runtime scenario due to simplicity of the ANN model they used.
The highlights of their work include:

\subsection{FEM Generated Temperature-Stress Dataset}
To best of our current knowledge, \cite{Zhang2016Fast} is the very first work that provide a
accuracy and substantial dataset of 3D IC temperature-stress info.
They have built a two-layer 3D IC model with $12\times12$ TSVs uniformly placed
in the whole chip using the FEM based software \textit{COMSOL}.
The size of whole chip is $1cm\times1cm\times300\mu m$, and it is
divided into $4\times4$ same sized blocks to represent $16$ cores,
both of the two layers are $1cm\times1cm\times100\mu m$. 
For the TSV structure, they set the values of $r_i$ and $r_o$
as $20\mu m$ and $24\mu m$, respectively. They also couple
the solid heat conduction field and the solid mechanical field.

Applying different thermal distribution to above model, they generated $2736$
pairs of temperature-stress data that can be used in neural network training.
The format of their data is like table \ref{tab::t-sformat}
\footnote{These data in tabular are randomly selected in dataset of \cite{Zhang2016Fast}}.
Note that the coordinate of $z$ is always $0$, because the dies in 3D IC are
extremely thin.

\begin{table}[htb]
\centering
\begin{tabularx}{30em}{*{3}{>{\centering\arraybackslash}X} *{2}{|c}}
    \toprule
    \multicolumn{3}{c|}{Coordinate($m$)} & 
    \multirow{2}{*}{Temperature($K$)} & 
    \multirow{2}{*}{Stress($N/m^2$)} \\
    \cmidrule{1-3}
    x & y & z && \\
    \midrule
    8.4631E-5 & 9.5289E-5 & 0 & 359.1351 & 31150.4105 \\
    2.0725E-4 & 1.2576E-4 & 0 & 359.0865 & 58634.6722 \\
    9.5824E-5 & 2.0547E-4 & 0 & 359.0865 & 62021.0153 \\
    \bottomrule
\end{tabularx}
\caption{Format of Temperature-Stress Dataset}
\label{tab::t-sformat}
\end{table}

\subsection{Hand-Crafted Feature Extraction}

\subsection{A Fast Fully-Connected ANN Model}

\section{Automatic Feature Extraction via CNN}
\subsection{A LeNet Based Model}
LeNet is a very early 
\subsection{A Tiny VGG-Net Based Model}

\section{CNN-Stress: an End-to-End Process}

\section{Experiments}
